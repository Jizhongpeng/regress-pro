\documentclass{article}
\usepackage{amsmath}

\title{Notes on Ellipsometry}
\author{Francesco}
\date{December 2016}

\begin{document}

\section{Measured Parameters}
Known formulas for ellipsometry in RPE configuration (rotating polarizer, fixed analyzer). The variable $A$ designate the analyzer's angle.

\begin{align}
    \alpha & = \frac{\tan^2 \Psi - \tan^2 A}{\tan^2 \Psi + \tan^2 A} \\
    \beta & = \frac{2 \tan \Psi \cos \Delta \tan A}{\tan^2 \Psi + \tan^2 A}
\end{align}

\begin{equation}
    \rho = \tan \Psi \, e^{i \Delta} = \frac{R_p}{R_s}
\end{equation}

\begin{align*}
     \tan^2 \Psi & = \frac{|R_p|^2}{|R_s|^2} \\
     \tan \Psi \cos \Delta & = \frac{\operatorname{Re}\{R_p R_s^\ast\}}{|R_s|^2} \\
     \cos \Delta & = \frac{\operatorname{Re}\{R_p R_s^\ast\}}{|R_s| |R_p|} \\
\end{align*}

\begin{align}
    \alpha & = \frac{|R_p|^2 - \tan^2 A \, |R_s|^2}{|R_p|^2 + \tan^2 A \, |R_s|^2} \\
    \beta & = \frac{2 \operatorname{Re}\{R_p R_s^\ast\} \tan A}{|R_p|^2 + \tan^2 A \, |R_s|^2}
\end{align}

\section{Useful identities}

Let $F(z) = U(z) + i V(z)$ be an analytic function. Let designate $z$ as $z = x + i y$. The derivative:
\begin{equation*}
    \frac{\partial |F|^2}{\partial x}
\end{equation*}

will be a real number. It will related to $\frac{\partial F}{\partial z}$ by the relation:
\begin{equation}
    \frac{\partial |F|^2}{\partial x} = 2 \operatorname{Re}\left\{F \frac{\partial F}{\partial z}^\ast\right\}
\end{equation}
the we use the relation:
\begin{equation*}
    \frac{\partial F}{\partial y} = i \frac{\partial F}{\partial z}
\end{equation*}
to obtain the derivative with $y$, the imaginary part of $z$:
\begin{equation}
    \frac{\partial |F|^2}{\partial y} = 2 \operatorname{Im}\left\{F \frac{\partial F}{\partial z}^\ast\right\}
\end{equation}
In the same spirit it follows:
\begin{equation}
    \frac{\partial \left( F G^\ast+G F^\ast \right)}{\partial x} = 2 \operatorname{Re}\left\{F \frac{\partial G}{\partial z}^\ast + G \frac{\partial F}{\partial z}^\ast\right\}
\end{equation}
and
\begin{equation}
    \frac{\partial \left( F G^\ast+G F^\ast\right)}{\partial y} = 2 \operatorname{Im}\left\{F \frac{\partial G}{\partial z}^\ast + G \frac{\partial F}{\partial z}^\ast\right\}
\end{equation}

The equation above can be used to obtain the derivative wrt the real and imaginary part of $n = n_r + i n_i$, the refractive index, of one of the medium.
\begin{equation*}
    \frac{\partial |R_s|^2}{\partial n_r} = 2 \operatorname{Re}\left\{R_s \frac{\partial R_s}{\partial n}^\ast\right\}
\end{equation*}
The same equation can be used to obtain the derivatives of $\rho$. The basis of the relation is the fact that $R_s$ and well as $R_p$ are analytical function of the $n$s of the mediums in the film stack.

\section{SE alpha beta derivatives from $R_s R_p$ squares}
Derivatives of SE $\alpha$ and $\beta$ with a generic parameter, called $x$ here. We consider $\alpha$ and $\beta$ as formal function of the quantities $|R_s|^2$, $|R_p|^2$ and $\operatorname{Re}\{R_p R_s^\ast\}$.
\begin{equation}
\frac{\partial \alpha}{\partial x} = \frac{2 \tan^2 A}{\left( |R_p|^2 + \tan^2 A \, |R_s|^2 \right)^2} \left( |R_s|^2 \frac{\partial |R_p|^2}{\partial x} - |R_p|^2 \frac{\partial |R_s|^2}{\partial x}\right)
\end{equation}
\begin{multline}
\frac{\partial \beta}{\partial x} = - \frac{2 \tan A \operatorname{Re}\{R_p R_s^\ast\}}{\left( |R_p|^2 + \tan^2 A \, |R_s|^2 \right)^2} \\ \left( \frac{\partial |R_p|^2}{\partial x} +  \tan^2 A \frac{\partial |R_s|^2}{\partial x}\right) + \frac{2 \tan A}{|R_p|^2 + \tan^2 A \, |R_s|^2} \frac{\partial \operatorname{Re}\{R_p R_s^\ast\}}{\partial x}
\end{multline}

Derivative with the analyzer's angle:

\begin{align}
\frac{\partial \alpha}{\partial A} & = - \frac{4 |R_s|^2 |R_p|^2 \tan A}{\left( |R_p|^2 + \tan^2 A \, |R_s|^2 \right)^2} \sec^2 A \\
\frac{\partial \beta}{\partial A} & = \frac{2 \operatorname{Re}\{R_p R_s^\ast\} \left(|R_p|^2 - \tan^2 A \, |R_s|^2\right)}{\left( |R_p|^2 + \tan^2 A \, |R_s|^2 \right)^2} \sec^2 A
\end{align}

\section{Useful Derivative Expressions}
\begin{equation*}
    \frac{\partial}{\partial x} \left( \frac{x-y}{x+y}\right) = \frac{2 y}{(x+y)^2}
\end{equation*}

\begin{equation*}
    \frac{\partial}{\partial y} \left( \frac{x-y}{x+y}\right) = - \frac{2 x}{(x+y)^2}
\end{equation*}

\end{document}
